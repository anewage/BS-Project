%% -!TEX root = AUTthesis.tex
% در این فایل، عنوان پایان‌نامه، مشخصات خود، متن تقدیمی‌، ستایش، سپاس‌گزاری و چکیده پایان‌نامه را به فارسی، وارد کنید.
% توجه داشته باشید که جدول حاوی مشخصات پروژه/پایان‌نامه/رساله و همچنین، مشخصات داخل آن، به طور خودکار، درج می‌شود.
%%%%%%%%%%%%%%%%%%%%%%%%%%%%%%%%%%%%
% دانشکده، آموزشکده و یا پژوهشکده  خود را وارد کنید
\faculty{دانشکده مهندسی کامپیوتر و فناوری اطلاعات}
% گرایش و گروه آموزشی خود را وارد کنید
\department{گرایش فناوری اطلاعات}
% عنوان پایان‌نامه را وارد کنید
\fatitle{
پیاده‌سازی ابزاری برای سنجش استفاده‌پذیری سامانه‌های مبتنی بر وب به روش جمع‌سپاری
}
% نام استاد(ان) راهنما را وارد کنید
\firstsupervisor{استاد احمد عبداله‌زاده بارفروش}
%\secondsupervisor{استاد راهنمای دوم}
% نام استاد(دان) مشاور را وارد کنید. چنانچه استاد مشاور ندارید، دستور پایین را غیرفعال کنید.
%\firstadvisor{نام کامل استاد مشاور}
%\secondadvisor{استاد مشاور دوم}
% نام نویسنده را وارد کنید
\name{امیر}
% نام خانوادگی نویسنده را وارد کنید
\surname{حقیقتی ملکی}
%%%%%%%%%%%%%%%%%%%%%%%%%%%%%%%%%%
\thesisdate{مرداد ۱۳۹۷}

% چکیده پایان‌نامه را وارد کنید
\fa-abstract{
با بررسی مدل‌های کیفیتی ارائه شده از سال ۱۹۷۰ تا به اکنون، با تقریب خوبی می‌توان گفت تمامی مدل‌های کیفی نرم‌افزار، استفاده‌پذیری را جزو مشخصه‌های اصلی کیفیت یک نرم‌افزار مطرح می‌کنند. وجه مشترک تعاریف متعددی که برای استفاده‌پذیری مطرح می‌شود، در سه بعد کاربر، انجام یک فعالیت مشخص و تعامل با یک واسط برای انجام آن فعالیت، قابل بیان است. به عنوان یک مهندس نرم‌افزار، افزایش کیفیت در محصولات و کاهش هزینه‌های ناشی از خرابی‌ها و یا درخواست‌های تغییر، چالشی تامل برانگیز است. وب‌اپلیکیشن‌ها به عنوان نوعی محصول نرم‌افزاری که در آن‌ها زیبایی، واسط کاربری و نحوه تعامل کاربران مهم است، به دلیل استفاده گسترده‌شان، می‌توانند تاثیر شگرفی در موفقیت یک پروژه صنعتی، کسب‌وکارهای نوپا و یا تسهیل زندگی روزمره با استفاده از نرم‌افزارها داشته باشند. از جمله نقاط ضعف بیشتر وب‌اپلیکیشن‌ها، طراحی نه‌چندان کاربرپسندانه واسط کاربری آن‌هاست که موجب شده تا در بسیاری از موارد، کاربران، علاقه‌مندی استفاده از محصول مبتنی وب یک سازمان را در عین سرمایه‌گذاری‌های زیاد آن سازمان برای جذب کاربر، از دست بدهند و در نتیجه متضرر شوند. گرچه، به صورت ایده‌آل، تمامی تصمیم‌گیری‌های مدیریتی و کلان (از قبیل اتخاذ مدل‌های فرایندی مناسب برای تولید نرم‌افزار با هزینه کم) با نهایت دقت و تجربه انجام می‌شوند، ولی در بسیاری از موارد همچون پروژه تقویم شرکت گوگل، مواردی ملاحظه می‌شود که واسط کاربری ناکارآمد، به ناچار، هزینه‌های گاهاً زیادی به تیم مهندسی نرم‌افزار تحمیل کرده است. با مروری بر منابع مختلف، ارزیابی و تست روی نمونه‌های اولیه رابط کاربری وب‌اپلیکیشن‌ها به منظور رفع نواقص آن‌ها، امری واضح به نظر می‌رسد. اما پاسخ دادن به این سوال که «چه واسط کاربری‌ای خوب است؟» همیشه آسان نبوده و با تغییر فناوری و گذشت زمان شاهد تغییر سریع در نیازمندی‌ها هستیم که شاید چک‌لیست‌ها و توصیه‌ها نیز پاسخگوی دقیقی برای آن‌ها نباشند. بررسی مدل‌های کیفیتی مختلف از سال ۱۹۷۰ تا به امروز و مقایسه تطبیقی آن‌ها، نشان می‌دهد که در جهت افزایش استفاده پذیری، خصیصه‌های مهمی مطرح شده‌اند که از جمله آن‌ها خصیصه‌های مطرح شده در سال ۲۰۱۳ بود. با در نظر داشتن این خصیصه‌ها ابزارهای موجود را مورد بررسی قرار دادیم و متوجه شدیم که تقریبا تمامی ابزارهای مطرح، تنها بخشی از این خصیصه‌ها را استفاده می‌کنند. در نهایت به پیاده‌سازی ابزاری به جهت تست و سنجش استفاده‌پذیری پرداختیم که به کاربر سامانه، امکان تست و سنجش استفاده‌پذیری طراحی سامانه مبتنی بر وب خود را می‌دهد و از سکوهای جمع‌سپاری نیز به منظور دستیابی به داده‌ها استفاده می‌کند.
}


% کلمات کلیدی پایان‌نامه را وارد کنید
\keywords{استفاده‌پذیری، مدل‌های کیفیتی، جمع‌سپاری، سامانه‌های کاربردی مبتنی بر وب}



\AUTtitle
%%%%%%%%%%%%%%%%%%%%%%%%%%%%%%%%%%