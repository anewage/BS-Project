%% -!TEX root = AUTthesis.tex
% در این فایل، عنوان پایان‌نامه، مشخصات خود، متن تقدیمی‌، ستایش، سپاس‌گزاری و چکیده پایان‌نامه را به فارسی، وارد کنید.
% توجه داشته باشید که جدول حاوی مشخصات پروژه/پایان‌نامه/رساله و همچنین، مشخصات داخل آن، به طور خودکار، درج می‌شود.
%%%%%%%%%%%%%%%%%%%%%%%%%%%%%%%%%%%%
% دانشکده، آموزشکده و یا پژوهشکده  خود را وارد کنید
\faculty{دانشکده مهندسی کامپیوتر و فناوری اطلاعات}
% گرایش و گروه آموزشی خود را وارد کنید
\department{گرایش فناوری اطلاعات}
% عنوان پایان‌نامه را وارد کنید
\fatitle{
	مروری بر پروژه کارشناسی - پیاده‌سازی ابزار مبتنی وب به منظور سجنش استفاده‌پذیری رابط کاربری سامانه‌های مبتنی بر وب به روش جمع‌سپاری
}
% نام استاد(ان) راهنما را وارد کنید
\firstsupervisor{استاد احمد عبداله‌زاده بارفروش}
%\secondsupervisor{استاد راهنمای دوم}
% نام استاد(دان) مشاور را وارد کنید. چنانچه استاد مشاور ندارید، دستور پایین را غیرفعال کنید.
%\firstadvisor{نام کامل استاد مشاور}
%\secondadvisor{استاد مشاور دوم}
% نام نویسنده را وارد کنید
\name{امیر}
% نام خانوادگی نویسنده را وارد کنید
\surname{حقیقتی ملکی}
%%%%%%%%%%%%%%%%%%%%%%%%%%%%%%%%%%
\thesisdate{تیر ۱۳۹۷}

% چکیده پایان‌نامه را وارد کنید
\fa-abstract{
	با تقریب خوبی می‌توان گفت تمامی مدل‌های کیفی نرم‌افزار، استفاده‌پذیری را جزو مشخصه‌های اصلی کیفیت یک نرم‌افزار مطرح می‌کنند. وجه مشترک تعاریف متعددی که برای استفاده‌پذیری مطرح می‌شود، در سه بعد کاربر، انجام یک فعالیت مشخص و تعامل با یک واسط برای انجام آن فعالیت، قابل بیان است. به عنوان یک مهندس نرم‌افزار، افزایش کیفیت در محصولات و کاهش هزینه‌های ناشی از خرابی‌ها و یا درخواست‌های تغییر، چالشی تامل برانگیز است. وب‌اپلیکیشن‌ها به عنوان نوعی محصول نرم‌افزاری که در آن‌ها زیبایی، واسط کاربری و نحوه تعامل کاربران مهم است، به دلیل استفاده گسترده‌شان، می‌توانند تاثیر شگرفی در موفقیت یک پروژه صنعتی، کسب‌وکارهای نوپا و یا تسهیل زندگی روزمره با استفاده از نرم‌افزارها داشته باشند. از جمله نقاط ضعف بیشتر وب‌اپلیکیشن‌ها، طراحی نه‌چندان کاربرپسندانه واسط کاربری آن‌هاست که موجب شده تا در بسیاری از موارد، کاربران، علاقه‌مندی استفاده از محصول مبتنی وب یک سازمان را در عین سرمایه‌گذاری‌های زیاد آن سازمان برای جذب کاربر، از دست بدهند و در نتیجه متضرر شوند. گرچه، به صورت ایده‌آل، تمامی تصمیم‌گیری‌های مدیریتی و کلان (از قبیل اتخاذ مدل‌های فرایندی مناسب برای تولید نرم‌افزار با هزینه کم) با نهایت دقت و تجربه انجام می‌شوند، ولی در بسیاری از موارد همچون پروژه تقویم شرکت گوگل، مواردی ملاحظه می‌شود که واسط کاربری ناکارآمد، به ناچار، هزینه‌های گاهاً زیادی به تیم مهندسی نرم‌افزار تحمیل کرده است. با مروری بر منابع مختلف، ارزیابی و تست روی نمونه‌های اولیه رابط کاربری وب‌اپلیکیشن‌ها به منظور رفع نواقص آن‌ها، امری واضح به نظر می‌رسد. اما پاسخ دادن به این سوال که «چه واسط کاربری‌ای خوب است؟» همیشه آسان نبوده و با تغییر فناوری و گذشت زمان شاهد تغییر سریع در نیازمندی‌ها هستیم که شاید چک‌لیست‌ها و توصیه‌ها نیز پاسخگوی دقیقی برای آن‌ها نباشند. بنابراین می‌بایست در طراحی واسط کاربری، به یک روش کمی و قابل استناد، نیازمندی‌ها را با استفاده از نمونه‌های اولیه بسنجیم (که به دلیل هنری انگاشتن اکثر کارها، این امر نادیده گرفته می‌شود). اما سنجش دقیق، نیازمند جمع‌آوری داده از ارزیابی و تست واسط کاربری توسط کاربران نهایی است تا بتوان تحلیل دقیق انجام داد و مشکلات طراحی واسط را به درستی تشخیص داد. یکی از روش‌های جمع‌آوری داده، استفاده از جمع‌سپاری است. باید توجه داشت که استفاده از جمع‌سپاری چالش‌هایی را فرارویمان خواهد گذاشت که از جمله آن‌ها می‌توان به عدم وجود صحت در داده‌ها اشاره کرد. در این پروژه وب‌اپلیکیشنی به منظور ارائه داشبورد مدیریتی برای صاحبان طراحی و افراد متمایل به انجام تست‌های مختلف با معیارهای متفاوت و دلخواه، پیاده خواهد شد. همچنین دادگان و پاسخ‌ها و تحلیل‌های تست اپلیکیشن در مواجهه کاربران واقعی با آن‌ها، به اطلاع کاربر خواهد رسید؛ علاوه بر موارد فوق، قسمت اصلی این پروژه در پاسخ به چالش صحت داده در روش جمع‌سپاری، ابتدا رفتار کاربران پاسخ‌دهنده (کارگران) توسط ماتریسی مدل می‌شود که برای مدل‌سازی و به دست آوردن مقادیر مدل‌ها، از روش تزریق سوالات طلایی استفاده خواهد شد. سپس در صورت پایین بودن کیفیت کار کارگران از حد مشخصی که در هنگام مدل‌سازی مشخص می‌شود، نتیجه کار آن‌ها به عنوان داده نامربوط شناخته شده و حذف می‌گردد. امکان تعریف تست‌های دلخواه و محدود نبودن به تست‌های از پیش تعریف شده تفاوت عمده ابزار استفاده‌پذیر با سایر ابزارهای مشابه است؛ از جمله ابزارهای مطرح موفق در این حوزه، می‌توان به UsabilityHub، Optimizely و CrazyEgg اشاره کرد که همانطور که ذکر شد، در طی این پروژه، سعی بر برطرف‌سازی برخی از نواقص آن‌هاست.
}


% کلمات کلیدی پایان‌نامه را وارد کنید
\keywords{کلیدواژه اول، ...، کلیدواژه پنجم (نوشتن سه تا پنج واژه کلیدی ضروری است)}



\AUTtitle
%%%%%%%%%%%%%%%%%%%%%%%%%%%%%%%%%%