\chapter*{‌پیوست}
\markboth{پیوست}{}
\addcontentsline{toc}{chapter}{پیوست}
موضوعات مرتبط با متن گزارش پایان نامه كه در يكی از گروه‌های زير قرار می‌گيرد، در بخش پيوست‌ها آورده شوند:
\begin{enumerate}
\item  اثبات های رياضی يا عمليات رياضی طولانی‌.‌
\item داده و اطلاعات نمونه (های) مورد مطالعه (\lr{Case Study}) چنانچه طولانی باشد‌.‌
\item نتايج كارهای ديگران چنانچه نياز به تفصيل باشد‌.‌
\item مجموعه تعاريف متغيرها و پارامترها، چنانچه طولانی بوده و در متن به انجام نرسيده باشد‌.‌
\end{enumerate}
% براي شماره‌گذاري روابط، جداول و اشكال موجود در پيوست‌ از ساختار متفاوتي نسبت به متن اصلي استفاده مي‌شود كه در زير به‌عنوان نمونه نمايش داده شده‌است. 
% \begin{equation}
%F=ma
%\end{equation}
\section*{کد میپل }
\begin{latin}
\begin{verbatim}

with(DifferentialGeometry):
with(Tensor):
DGsetup([x, y, z], M)
																	frame name: M
a := evalDG(D_x)
																	D_x
b := evalDG(-2 y z D_x+2 x D_y/z^3-D_z/z^2)


\end{verbatim}
\end{latin}